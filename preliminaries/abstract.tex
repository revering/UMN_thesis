%%%%%%%%%%%%%%%%%%%%%%%%%%%%%%%%%%%%%%%%%%%%%%%%%%%%%%%%%%%%%%%%%%%%%%%%%%%%%%%%
% abstract.tex: Abstract
%%%%%%%%%%%%%%%%%%%%%%%%%%%%%%%%%%%%%%%%%%%%%%%%%%%%%%%%%%%%%%%%%%%%%%%%%%%%%%%%
While strong evidence for the existence of dark matter has been found in astronomical observations and simulations of galaxy structure formation and cosmic baryogenesis, no repeatable observations have been made of its potential non-gravitational interactions. 
The high energy interactions within particle accelerators present an environment where these dark matter interactions may be produced and their existence inferred from visible signatures in detectors. 
Traditional searches for collider-based dark matter require either visible final states or additional high-energy recoil particles to tag potential dark matter interactions, greatly lowering their selection efficiency. 
In this work I present a search for a dark-matter-like signal in proton-proton collisions as a center of mass energy of 13 TeV using the CMS detector at the CERN LHC, using tag-and-probe techniques to select for interactions of standard-model muons with the detector material.
For dark matter models without visible decay products, these interactions are characterized by muon tracks which `disappear', losing large amounts of energy without associated deposits in the calorimeters.
The analysis is performed on the data set collected using the CMS experiment during 2018, corresponding to an integrated luminosity of 60.2$fb^{-1}$.
By using CMS as a fixed target experiment the inefficiency of selecting non-visible signatures in the initial collision can be avoided, extending the CMS sensitivity to dark matter models with hidden-sector decays or long lifetimes.

%%%%%%%%%%%%%%%%%%%%%%%%%%%%%%%%%%%%%%%%%%%%%%%%%%%%%%%%%%%%%%%%%%%%%%%%%%%%%%%%
