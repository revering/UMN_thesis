%%%%%%%%%%%%%%%%%%%%%%%%%%%%%%%%%%%%%%%%%%%%%%%%%%%%%%%%%%%%%%%%%%%%%%%%%%%%%%%%
% backgrounds.tex
%%%%%%%%%%%%%%%%%%%%%%%%%%%%%%%%%%%%%%%%%%%%%%%%%%%%%%%%%%%%%%%%%%%%%%%%%%%%%%%%
\chapter{Backgrounds}
\label{backgrounds}
There are several ways that standard model physics processes could mimic signal events, and the expected rate of these processes must be understood in order to meaningfully interpret the number of observed events in the signal regions in data.
Wherever possible, this rate is determined using control regions in data, where orthogonal selections to the signal region are applied in order to find regions enriched in backgrounds to study their properties and extrapolate their rate to the final selection.

Heuristically, the backgrounds in this search are split into two types: those where the probe track is produced by a muon ("muon backgrounds") and those where it is produced by any other particle in the event ($\mu+X$ events).
In practice, the invariant mass of the tag/probe pair is the primary feature used to distinguish events with non-muon probes, as muon-system and calorimeter information are heavily used in the signal definition and therefore unavailable for simple definitions of control regions.
As a result, the $\mu+X$ backgrounds include any processes which do not have a peak near the Z-mass, effectively serving as a background of events without muons from Z decays.
While these muons from other sources could potentially produce signal events in the same manner as muons from Z bosons, the much smaller relative rate causes this choice to have very little impact on the resulting sensitivity.

\subsection{Muon Backgrounds}
Standard model DY backgrounds occur when the base event selection successfully finds a probe track which matches to a muon originating from a Z boson which then muon passes the signal selection without undergoing a Dark Bremsstrahlung. 
This can occur either when muons lose energy via conventional physics interactions without producing enough visible energy in the detector to reject the event or when the muon signature in HE and the CSC chambers is missed.
Often, some combination of both effects is present as muons with altered energy from interactions within the detector can have genuine deflections causing them to pass through fewer stations in the muon chambers, greatly increasing the probability of missing their signatures.

The primary difficulty of muon backgrounds is the lack of control regions enriched in muon backgrounds without containing significant signal contamination.
Like signal events, muon backgrounds have opposite-sign pairs, and have nearly identical distributions of invariant mass, track isolation, and muon kinematics.
Muon backgrounds can have slightly higher HCAL isolation than signal events, as muons with large energy loss to SM particles have lower energies and therefore reduced cross sections for \dbrem, but these high energy scatters produce large differences in the muon chamber response.

Due to this difficulty in defining signal regions without significant signal contamination or large differences from the signal selection, we predict the muon background rate for both regions using MC and perform several validation studies to confirm that the features used to define missed and deflected muons in simulation closely match those seen in data.




//Split muon background sources into high-E scatters and missed muons
We control the muon interaction backgrounds through study of data control regions and simulated DY containing large energy deposits in HCAL, as well as specially simulated DY with missing CSC hits required. The location-dependent rate of missing CSC hits in each depth is corrected using the methods described in Section \ref{sec:missingCSC}.
As these backgrounds can be produced through a combination of muon energy loss and poor muon reconstruction, the predicted backgrounds are made using a combination of both studies. In the complete disappearance region, the validated DY simulation is used with the missing CSC hit studies to produce the number of expected missed muons. In the partial disappearance region, the DY MC is used to train the BDT and predict the expected number of DY background events, with corrections and uncertainties from these validations and control regions applied.
\subsubsection{Missed Muons}

Events with a muon matched to the probe track may pass the complete disappearance selections by having the muon fail to produce hits in any CSC station due to poor reconstruction. This can potentially be caused by inefficient CSC stations, gaps in the detector, or physics process that deflect the muon sufficiently to pass the nearby CSC hit requirements without depositing enough energy in the calorimeters to fail isolation requirements. This missed muon background is predicted using simulation, with an additional correction factor applied using studies of CSC hit reconstruction efficiency in data. The rate of missed muons depends only on the kinematics of the selected track, and is otherwise expected to be independent of the initial process if there are no fake muons chosen. As the DY process produces the vast majority of selected muons, only simulated events produced from Z decays are considered.

To efficiently simulate these events, DY events are generated with a filter requiring at most one nearby CSC hit. An additional correction is applied to account for CSC regions that have worse performance in data by counting events with only one nearby CSC hit in those regions as having zero, with event weights proportional to the difference in CSC segment reconstruction efficiency (See Section \ref{sec:missingCSC}). Using this simulation, 4.6$\pm$3 missed muon events are predicted in the complete disappearance category for the full 59.8$\fbinv$ of Run 2. This background prediction accounts for backgrounds both from failed reconstruction and hard Brem or other scattering processes, and relies on the accuracy of the muon scattering in simulation as validated in Section \ref{sec:HardBrem}.

\subsection{Non-Muon Backgrounds}
Non-Muon ("$\mu+X$") events occur when a tagging muon is paired with a probe track that does not originate from the decay of a Z boson. 
Non-Muon tracks are very likely to be decay within or be stopped by the calorimeters, mimicking signal events by having little or no ionization in the muon chambers. 
Unlike signal events, the interactions of these backgrounds in the calorimeters should produce visible energy deposits which can be used to reject the event.
Most importantly, $\mu+X$ events will not have a sharp peak in the tag/probe invariant mass like signal and DY events will.
Because of this, control regions with inverted invariant mass selections can be defined which are composed nearly entirely of $\mu+X$ events, allowing for an entirely data-driven estimation of the background rate.

Among the possible particles which may produce the misidentified probe tracks, of particular interest are charged mesons such as pions or kaons, which may decay into muons as they pass through the detector.
As the neutrinos produced in these decays are invisible to the detector, charged mesons decaying into muons could appear very similar to a signal event, with the misidentified meson acting as the initial muon and the muon produced in the decay acting as a deflected muon with energy loss to a neutrino instead of a dark photon.
The overall rate of these meson backgrounds is reduced by several factors.
First, the initial meson must decay within the detector, and must do so before undergoing significant showering in the detector.
Secondly, the low mass of the neutrino cause favor the muon carrying most of the outgoing momentum in the collision, so that only a small fraction of meson decays result in significant energy differences between the muon and the initial meson.
Lastly, the mesons produced in the collision are rarely isolated, and are generally produced in jets of particles which are likely to cause the event to fail the isolation selections.

While $\mu+X$ backgrounds specifically refer to events where the selected probe track does not match to a muon produced in the DY process, DY$\rightarrow\mu\mu$ events can contribute to this region when one muon is reconstructed well and is used as the tag while the other is out of detector acceptance or is reconstructed poorly. 
In these cases, an un-associated track may be selected as the probe and a non-muon background can be produced from a DY$\rightarrow\mu\mu$ event. 
This corresponds to roughly 5$\%$ of the expected $\mu$+X background.

The frequency of $\mu+X$ events is predicted via data driven methods using control regions of events with invariant masses far from the Z-peak or with same-sign tag/probe pairs.
In the complete disappearance region, the off-peak events are fit with a non-peaking function to predict the background in the signal region, and the same-sign control region is used to validate our assumptions of the background shape and non-correlated charge.
In the partial disappearance region, the DY contribution to the off-peak region is too large to make an effective fit, and so the same-sign control region is used to determine the BDT score of $\mu$+X events and directly predict the background rate.

\subsubsection{Off Peak Control Region}
In the off-peak control region, the full total disappearance selections are applied with invariant mass requirement inverted to be far from the Z-peak ($M_{\mu\mu}$ $<70$ or $>110\GeV$). The resulting region contains very little DY contribution due to the invariant mass and total disappearance requirement and so contains almost entirely $\mu+X$ background. By fitting the off-peak region using a non-peaking background function (Eq. \ref{eq:bkgfunc}) derived using a control region of poorly isolated probe tracks (Appendix \ref{sec:muPlusX}), the expected rate in the signal region can be predicted entirely from data. While the complete signal selection does not contain a large enough sample size in MC to compare to data, by removing the HCAL isolation requirements the rates in data and MC are increased enough to test the chosen functional form and verify that the observed rate is close to the most significant expected backgrounds. Good agreement between data and MC is seen in this subset of events (Fig. \ref{fig:offpeakfit}, left).

Due to the small number of events in the off-peak control region after fully applying the HCAL isolation requirements, we apply Kernel Density Estimated smoothing to the data points to avoid bias from bins with zero events in the fit (Fig. \ref{fig:offpeakfit}, right). Two separate smoothings are done for events below and above the excluded region, with tuning factors of 5 and 1.2, respectively. The results of this fit are used to predict the $\mu+$X background in the complete disappearance region, but are dependent on the assumption of a smooth functional form. Two separate fits are performed - one directly using the background function on the smoothed data, and one by first fitting data events in the non-isolated region and then re-scaling the result to the isolated region. The re-scaled fit has less statistical uncertainty due to the larger sample size in the non-isolated region, but does not include any potential change in shape due to applying the isolation requirements. The free fit predicts 6$\pm$3 background events, while the re-scaled non-isolated fit predicts 4.9$\pm$0.3 events. The re-scaled fit is taken as the central value of the predicted yield, and the difference between the two is taken as a systematic uncertainty on the choice of fitting function.
subsubsection{Non-Muonic Z decays}
The most significant expected background that might produce an invariant mass peak in within the signal region comes from non-muonic Z decays, such as Z to $\tau\tau$ events where one $\tau$ decays leptonically and produces a good tag muon, while the other might decay into a pion or other particle which fakes a probe muon. To study these events, DY events which match the selected probes to gen level $\tau$ particles were studied to confirm that they did not produce an invariant mass peak within the signal region and it was observed that they have a smoothly falling distribution which will be included by the existing fit (Fig. \ref{fig:ZtoTauTau}).
MC samples of hadronic Z decays were also analyzed to determine whether they would be sufficiently covered by the off-peak control region. Because of the very low rate of hadronic Z MC events passing the probe isolation requirements, the invariant mass distribution was measured without applying track or ECAL isolation requirements (Fig. \ref{fig:ZtoTauTau}). As observed in Z to $\tau\tau$ events, the additional particles produced in the hadronic decays of the Z boson shift the invariant mass of the tag and probe such that there is no longer a peak near the Z mass and these potential background sources will be included by the off-peak control region.

\begin{equation} \label{eq:bkgfunc}
f(s)=A(s-B)e^{-Cs} 
\end{equation}
\subsubsection{Same Sign Control Region}
\label{sec:sameSign}
As the most significant non-muon probe backgrounds are expected to be produced by processes without charge-correlated tags and probes, the rate of these events with same-sign pairs should be very close to the rate with opposite-sign pairs, and a same-sign control region can be defined to verify that and cross-check the expected background rates. The same-sign control region is defined with the tag and probe charge requirement inverted and otherwise identical selections for both the complete and partial disappearance regions. Because of the close similarity to the signal region, the BDT scores of partial disappearance background events in this control region can be used to predict the expected background rate from $\mu+$X events. In the current skim 14.73 $\fbinv$ of 2018 data was available and one same-sign event passed the partial region selection. That event has a BDT score of $<$0.001. In Figure \ref{fig:sameSign}, the same-sign events in the complete disappearance region are plotted along with the fit function from the off-peak control region scaled to the same luminosity. The expected same-sign rate is within uncertainty of the measured rate, indicating that there is not an excess of opposite-sign events and that the assumption that the background is primarily from non-charge-correlated pairs is valid.
\subsection{Signal Contamination}
\label{sec:SigContamination}
A table of the relative rates of signal in each control region to each signal region for all mass points is presented in Table \ref{tab:SigCont}.
In the Off-peak control region, signal contamination would increase the predicted $\mu+X$ background by being included in the non-peaking fit applied to the control region. Using the functional form of the off-peak fit used in Figure \ref{fig:offpeakfit}, each signal event in the off-peak control region will add 0.5 expected $\mu+X$ events. Because of the strongly peaking nature of the signal events in invariant mass, signal contamination would reduce the observed by at most $2.5\%$ for all signal masses and regions.

While the off-peak signal contamination is small enough that it can be used to predict the rate of background $\mu$+X events, it is significant enough to prevent the use of the region as validation for the partial disappearance region. The signal contamination rate is almost entirely produced by the peaking shape of the invariant mass distribution, and any corrections or uncertainties derived using off-peak partial disappearance resulting from signal contamination would scale with the peaking DY shape instead of the smooth $\mu$+X background, resulting in a loss of signal near 100$\%$.

For the same-sign control region, signal contamination would produce a peak in the invariant mass distribution reducing the quality of the fit. No MC signal events passed the signal selection with a same-sign requirement for the tag muon and probe track, and no excess of same-sign events is observed near the Z peak in data.

In the hard Bremsstrahlung control region, signal contamination could result in shifts in the expected rate of Standard Model Bremsstrahlung and increased uncertainty in the measurement of HE energy deposits. Signal events are less likely to have hard Bremsstrahlung events before producing dark matter due to the dependence of the production cross section on muon energy, and less likely to deposit large amounts of energy in HE after a dark matter interaction due to reduced available muon energy. In addition, DY events in the hard Bremsstrahlung control region appear much more background-like than in the signal region due to energy loss from SM processes. As a result, background events pass the requirement of BDT score $>0.98$ much more frequently in the hard Bremmstrahlung control region than the signal region, while signal events have similar BDT score distributions in both regions. The combination of these effects results in an expected signal loss of 4$\%$ from signal contamination in the corrections derived in the hard Bremsstrahlung region.
