%%%%%%%%%%%%%%%%%%%%%%%%%%%%%%%%%%%%%%%%%%%%%%%%%%%%%%%%%%%%%%%%%%%%%%%%%%%%%%%%
%theory.tex: Chapter on light dark matter theory
%%%%%%%%%%%%%%%%%%%%%%%%%%%%%%%%%%%%%%%%%%%%%%%%%%%%%%%%%%%%%%%%%%%%%%%%%%%%%%%%
\chapter{Theoretical Framework}
\label{theory}
%%%%%%%%%%%%%%%%%%%%%%%%%%%%%%%%%%%%%%%%%%%%%%%%%%%%%%%%%%%%%%%%%%%%%%%%%%%%%%%%
There are two key aspects to the theory used in the search presented here: a model of a new physics which could produce dark matter used to simulate its interactions and determine our sensitivity, and a model of known physics, necessary to understand conventional processes which could mimic dark matter interactions as well as describe the initial states produced in the collision and the response of the detector to any produced particles.

The primary framework used to describe currently known particle physics is the standard model (SM), a quantum field theory which describes the set of known fundamental particles and their interactions.
In classical theories, particle interactions can be described via forces proportional to fields, such as the electromagnetic field applying a force directly proportional to the charge of a given particle.
In quantum field theories, particle interactions are instead described through the exchange of a gauge boson, allowing them to satisfy the requirements of special relativity and quantum physics. 
The SM contains five gauge bosons, which describe interactions with the nuclear strong, electroweak, and Higgs fields.
In addition, the SM includes six quarks, colored particles which can interact via the strong force, and six leptons, particles which only interact via the electroweak force.
A graphical representation of these fundamental particles is shown in \Cref{fig:SM}, along with numerical values of some of their physical properties.

\begin{figure}[htpb]
	\includegraphics[width=0.9\textwidth]{figures/smParticles.pdf}
	\centering
	\caption{Fundamental particles within the standard model.}
	\label{fig:SM}
\end{figure}

\begin{figure}[htpb]
	\includegraphics[width=0.9\textwidth]{figures/smRatePreds.pdf}
	\centering
	\caption[CMS Cross Section Results]{Predicted and measured cross sections over several sectors of CMS physics analyses. The predictions of the SM match the observed interaction rates to high precision over many orders of magnitude in cross section and a broad spectrum of processes.}
	\label{fig:SMratePreds}
\end{figure}

Developed over several decades through close collaboration between theory and experiment, the SM has been extremely successful at predicting a wide range of observed physics phenomena \Cref{fig:SMratePreds}.
Despite this success, there are several missing pieces in the SM such as gravity, neutrino masses, dark energy, and most significant to this paper, dark matter.
While these phenomena are observed in experiments, the lack of full information about the nature of the particles themselves or their interaction prevent their inclusion in the SM, and so keep it from being a complete description of all observed particle interactions.  

\section{Particulate Dark Matter}

%Graviational Lensing / Bullet Cluster

%Cosmic Microwave Background
Some of the most detailed measurements of the relative abundance of dark matter come from observations of the cosmic microwave background (CMB).
The CMB is radiation which originated in the early universe, when SM matter consisted of hydrogen and helium plasma and free electrons.
Because of scattering with these free electrons, the universe was opaque on cosmological scales until the plasma cooled and was converted into a gas. 
At this point, thermal radiation from the plasma could travel through the universe without significant scattering probability, and these photons can be observed today as the CMB, constant microwave radiation that is visible from every direction \cite{kurkisuonio} (\Cref{fig:CMB}).

\begin{figure}
	\label{fig:CMB}
	\centering
	\includegraphics[width=0.7\textwidth]{figures/cmb.png}
	\caption[The CMB temperature anisotropy]{\cite{PlanckCMB}.}
\end{figure}

The conversion of primordial plasma into a hydrogen and helium gas occurred when the universe was roughly 400,000 years old, and the CMB therefore provides an image of its temperature and density distribution at that time.
While the primordial plasma was nearly homogeneous, there were slight variations of order one part in ten thousand, which later grew by gravitational attraction and led to the formation of galaxies and stars \cite{Hurki-Suonio}.
By studying these anisotropies, the dynamics of the baryon-photon plasma can be inferred, and several properties of early cosmology, including the mass density of dark matter, can be determined.

The anisotropy of the CMB is measured through correlations in the temperature of different points as a function of the length scale between them. 
As the CMB forms a 'sphere' of the observable sky, these distance are expressed as length scales, and it is convienient to analyze them as expansions of spherical harmonics, which can be shown to only depend on the order of the angular scale parameter $l$. 
By measuring the pertubation of the CMB and decomposing it into sperical harmonics, the angular power spectrum can be measured \Cref{fig:CMBpowerSpectrum}, from which details about the composition of the early universe can be inferred.

The angular power spectrum of the CMB consists of a series of peaks of decreasing amplitude with increasing multiple moments. 
At large scales, $\theta >> $\ang{1}, the distances between spatial points are larger than the Hubble distance between them, and therefore cannot have any dynamic evolution in the primordial plasma.
At smaller scales, $\theta < $\ang{1}, the perturbations are close enough to communicate before the plasma condenses, and so can show evidence of plasma dynamics.

\begin{figure}
	\label{fig:CMBpowerSpectrum}
	\centering
	\includegraphics[width=0.7\textwidth]{figures/cmb_power_spectrum.png}
	\caption[The temperature angular power spectrum of the CMB]{\cite{PlanckCMB}.}
\end{figure}


Initially, anisotopies are present from quantum fluctuations which expanded to cosmological scales during inflation.
Regions with increased densities then attract the photon-baryon fluid gravitationally, amplifying the flucutation.
This increased density then causes a corresponding increase in the temperature of these regions and thus their radiation pressure, which eventually overcomes the gravitational attraction and begins to push the fluid out of the region. 
This pattern of gravitational and radiation driven motion form acoustic oscillations with Fourier modes that oscillate at frequencies dependant on the sound speed in the baryon-photon fluid.

The expansion of the universe causes the sound speed and plasma density to change with time, with the notable effect that the perturbation amplitude is maximal at angular scales which are at their extrema when the photons decoupled. 
This produces the peak structure seen in the CMB power spectra, with each maximum corresponding to an acoustic mode with frequency which happened to be at a maximum during the time of decoupling.

Beyond solely acoustic oscillations, this pattern is also strongly impacted by the presence of dark matter.
Initially distributed similarly to the photon-baryon plasma, dark matter will also be drawn into gravitational wells caused by regions of increased density produced by quantum flucutations before inflation.
Unlike SM matter, dark matter has very little or no radiation pressure, and thus will not directly undergo acoustic oscillations with the plasma. 
Only interacting via gravitational potential, dark matter follows the plasma into the gravitational wells but not leaving from the radiation pressure. 

Due to this interaction difference, the presence of dark matter significantly changes the oscillation frequency and the relative amplitude of each acoustic peak.
The angular power spectrum can thus be fit with a seven parameter cold dark matter model and produce strong constraints on the abundance of dark matter.
Through these fits, the mass density of dark matter in the universe is determined to be \SI{4.1e-27}{\kilo\gram\per\meter\squared} \cite{PlanckCMB}, roughly 5 times more than the total abundance of SM matter.

%Structure Formation

%Baryogenesis

\section{Thermal Dark Matter}
To help identify dark matter signatures that may be visible in a detector based experiment, several assumptions can be made relating to its nature. 
The primary one made here is a thermal origin for dark matter.
The core assumption of thermal dark matter models is that the interaction between dark matter and SM matter is significant enough that for some period of time in the early universe dark matter was in thermal equilibrium with standard model matter.
This allows dark matter to be produced simultaneously with SM matter during the universe's inception rather than as a separate process, and also requires that dark matter have some coupling to SM matter, which would also be necessary for its observation in an experiment like CMS.

This thermal equilibrium is achieved to due the high temperature and density of the early universe where, despite small SM couplings, dark matter particles can be frequently produced through interactions with standard model particles and produce standard model particles through their own interactions and decays.
As the universe expands and cools, the rate of these interactions decreases until the dark matter interaction rate fall below the Hubble expansion rate and "freezes out", leaving thermal equilibrium with SM matter and leaving its phase-space distribution subject only to the universe expansion and eventual structure formation \cite{thermalDM}.

For any given thermal dark matter model a 'relic target' cross section can be calculated - the cross section required to produce the density of dark matter seen today.
For any given cross section, the thermal relic model will set a specific freeze-out temperature and number density.
By combining this number density with the observed astronomical mass density, the thermal relic cross section can be related directly to a dark matter mass, with heavier particles requiring lower number densities and therefore later freeze-out temperatures and larger interaction cross sections.

\section{Light Dark Matter}
Traditional searches for thermal dark matter have tended to focus on models which interact via the nuclear weak force, which have relatively larger dark matter mass requirements (M$>\sim$few GeV), and so are easier to observe with indirect detection experiments searching for collisions of halo dark matter particles with passive detectors.
While this larger mediator mass, weakly interacting phase space is well-covered from a wide variety of experiments, dark matter could also be explained using mediators in a 'light' mass region, ~MeV to ~GeV, which act as a 'portal' to couple dark sector particles to the SM \cite{darkSectors}.

A simple renormalizable interaction which can be introduced to the SM which could create this light coupling is a kinetic mixing between a new gauge boson with field strength $F'_{\mu\nu}$ and the hypercharge field $B^{\mu\nu}$ through the operator $\mathcal{L}$ (\Cref{eq:LDMlagrangian}). 

\begin{equation}
	\label{eq:LDMlagrangian}
	\mathcal{L} = - \frac{\epsilon}{2} B^{\mu\nu}F'_{\mu\nu}
\end{equation}

For small mixing parameter $\epsilon$ and light gauge bosons, these new couplings align with those of the SM photon \cite{Bauer_2018} and so this new gauge boson is referred to as the "dark photon", or A'.
Through these couplings, any SM process which produces a photon could instead produce a dark photon, with suppression proportional to the mixing parameter.

This coupling represents the minimal kinetic mixing, encoding the kinetic coupling as a free parameter instead of arising from specific interactions within a chosen theory. 
While technically forming a dark sector on its own, this model must be extended with additional particles to act as dark matter candidates, as models with only dark photons would cause them to decay back into SM particles.
These dark matter candidates would couple to the dark photon via dark-sector gauge interactions with some coupling strength $g_D$, and can be either fermions or scalar bosons.
As these dark sector particles are assumed to not leave visible signatures within our detector, their influence in this search is limited to setting the branching fraction of the dark photon decaying to either visible SM particles or 'invisible' dark matter.
As this search focuses on these invisible dark photon signatures, it is assumed that $g_D$ is much larger than $\epsilon$ and that the mass of the dark matter candidates is less than twice the A' mass such that they will decay dominantly to dark matter particles or have significantly long lifetimes to pass out of the detector acceptance.


\section{Targeting Dark Matter Production}
If a light dark matter coupling were present, there are several potential signatures that could appear in an experiment like CMS.
Traditional searches in CMS look for particle production in the initial scattering process between proton constituents.
As a light dark matter coupling would allow mixing between the \aprime and any standard model photon, dark matter could be produced in initial state collisions with electromagnetic interactions between partons that produce photons that then mix with the \aprime.

Unfortunately, decays of the \aprime outside of the detector or into other dark matter particles which are invisible to CMS make this initial production mode very difficult to distinguish.
Events with large initial state radiation from the partons can make this signal visible as a large missing transverse momentum in the event, but the large reduction in acceptance from this type of requirement causes the resulting limit to be very weak.

Instead of searching for dark matter production in the initial state, the signal could appear in a secondary interaction between a final state particle and the detector itself.
The relatively small masses of light dark matter particles limits the loss in the interaction rate due to the much lower center of mass energy in these secondary interactions compared to the initial collision, and the high particle flux and large size of the detector result in a significant fixed target luminosity.

The primary signature of a secondary interactions between a particle emitted from the initial collision and the detector would be missing energy carried by the invisible \aprime and a change in the trajectory of the visible particle.
These types of signatures are difficult to observe in particles with large standard model interaction rates, as the shower of particles produced in their interactions can lead to large uncertainties in their energy measurement and little hope of observing trajectory differences.
In addition, high interaction rates expose particles to less of the detector as they are stopped relatively quickly, reducing the effective luminosity.

For these reasons, the selections in this search focus on potential dark matter interactions of final state muons.
As muons have small standard model cross sections they lose very little energy while they traverse the detector, leading to high consistency in projecting their trajectory and observing potential changes, as well as low rates of stopped muons due to standard model processes which could fake the energy loss to an \aprime.

With a light dark matter mixing, muons would primarily produce dark matter in the mass ranges of interest via a process known as \dbrem, where a dark matter particle is emitted from a muon as it recoils from the electromagnetic field of a nucleus in the detector (\Cref{fig:dbrem_feyn}).
While other processes, such as photoproduction of vector mesons which then decay invisibly through dark matter coupling, would also be present with the assumed coupling, only \dbrem is considered in this search due to its significantly larger production rate.

\begin{figure}[ht]
	\centering
	\label{fig:dbrem_feyn}
	\includegraphics[width=\textwidth]{figures/dbrem_feyn_diagram.jpg}
        \caption[Dark Bremsstrahlung Feynman Diagram]{Dark Bremsstrahlung caused by a muon interacting with a nucleus with atomic number Z.}
\end{figure}

As an additional feature of muon-initiated \dbrem, results from this search are also sensitive to dark matter models which replace the generic dark matter mixing with an asymmetric coupling to the lepton generations.
While lepton universality is built into the standard model, no such requirement is necessary for new physics models. 
Instead of the fully generic kinetic mixing presented in \Cref{eq:LDMlagrangian}, the dark photon gauge field can be chosen to contain an additional current which carries charge proportional to muon number minus tau number, ($L_\mu - L_\tau$) \cite{neut_trident}.
This type of field would result in a dark matter coupling that preferentially interacts with muons and taus, and has few existing constraints as it only would affect neutrinos and unstable leptons.

There are several experimental anomalies which could potentially be explained by an asymmetric lepton coupling of this type.
The measurement of the muon magnetic moment by the Fermilab E989 experiment \cite{gminus2}, combined with the original result from Brookhaven \cite{gminus2_bnl} leads to a 4.2 $\sigma$ discrepancy with the theoretical prediction \cite{gminus2_theory}. 
This could potentially be explained by a dark matter signal which preferentially couples to muons, as the dark matter coupling would alter the muon magnetic moment through additional loop diagrams.
A muon-philic cross section would also have a larger thermal relic cross section, as requiring a muon or tau coupling in its production reduces the available paths in its creation and so requires increasing the remaining cross section to achieve the same freeze-out number density.

\section{Z-Bosons and the DY Process}
For muons to potentially interact within the CMS detector and produce dark photons, they must first be created within the initial collision.
Many processes in CMS collisions can produce outgoing muons, but while accepting muons from any process would allow for very high signal rate the difficulty of identifying a muon using only inner detector information (see \Cref{sec:muonReco}) would result in large contributions from backgrounds where the reconstructed particle does not originate from a real muon.
To help control these backgrounds, muons from particle resonances can be selected, where neutral particles created in the hard scatter between the initial state quarks produce opposite-sign pairs of muons.
Because they are produced from the decays of mediator particles, the invariant mass of the dimuon pairs produced in these resonances peaks sharply near the particle mass, with a width proportional to its lifetime. 
By using muons from particle resonances, one outgoing muon can be used to 'tag' the event, and the other can be used to 'probe' its response in the detector to search for non-standard interactions. 

While many such resonances exist in CMS (\Cref{fig:diMuonSpectrum}), this search targets muon originating the decay of Z bosons, due to their higher energy of the produced muons and the reduced rates of non-resonant muon pairs near the Z mass.
Z bosons are the neutral carrier of the nuclear weak force, which unlike the electromagnetic and nuclear strong forces has massive force carrying bosons.
Specifically, the Z boson mass is near \SI{90}{\giga\eV}, 
Z bosons will decay into dilepton pairs with nearly equal rates between electrons, muons, and taus. 

The Z boson resonance in dimuon pairs is primarily produced through the Drell-Yan (DY) process, in which a pair of quarks in the initial hadron collision annihilate to a Z boson which subsequently decays into leptons.
The typical invariant mass of dimuon pairs produced in the DY process with minimal selection requirements is shown in figure \ref{dmuonInvMass}, along with the typical muon energy in \ref{dimuonEnergy}.

\begin{figure}[ht]
	\label{fig:diMuonSpectrum}
	\centering
	\caption[Inclusive dimuon spectrum in CMS]{The dimuon invariant mass spectrum in the CMS high-level trigger. Each peak represents one or more particle resonances. Figure from \cite{cmsMuonPerformance}}
	\includegraphics[width=0.5\textwidth]{figures/cms_diMuonSpectrum.jpg}
\end{figure}


%%%%%%%%%%%%%%%%%%%%%%%%%%%%%%%%%%%%%%%%%%%%%%%%%%%%%%%%%%%%%%%%%%%%%%%%%%%%%%%%
