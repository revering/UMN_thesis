%%%%%%%%%%%%%%%%%%%%%%%%%%%%%%%%%%%%%%%%%%%%%%%%%%%%%%%%%%%%%%%%%%%%%%%%%%%%%%%%
% experiment.tex: The CMS detector
%%%%%%%%%%%%%%%%%%%%%%%%%%%%%%%%%%%%%%%%%%%%%%%%%%%%%%%%%%%%%%%%%%%%%%%%%%%%%%%%
\chapter{The CMS Detector}
\label{detector}
%%%%%%%%%%%%%%%%%%%%%%%%%%%%%%%%%%%%%%%%%%%%%%%%%%%%%%%%%%%%%%%%%%%%%%%%%%%%%%%%

The compact muon solenoid (CMS) detector is one of two general-purpose detectors located at the large hadron collider (LHC) near Geneva, Switzerland. The LHC consists of two counter-rotating proton beams in a 27 km ring which cross at several designated points, producing particle collisions with center-of-mass energies near \SI{13.7}{\tera\eV}. 
One of these crossing points is in the center of the CMS detector, where 'bunches' of $\thicksim10^{11}$ protons collide every \SI{25}{\nano\second} in a FIND NUMBER beam spot. 
On average, each bunch crossing has 32 proton-proton interactions, though most are 'pileup' interactions from proton scattering or other unwanted processes.
Occasionally, these collisions will produce exotic, high energy interactions between the quarks within the protons. 
The proton interactions produce large numbers of secondary particles which radiate outward from the beam spot, and the detector measures and identifies these particles in order to reconstruct the physics processes that occurred during the collisions. 
While the bunches interact at a rate of \SI{40}{\mega\hertz}, the large amount of data produced in each collision and the limited data transfer and storage ability limits the final readout to $\thicksim$\SI{1000}{\hertz}.
To obtain this dramatic reduction in event rate, a triggering system is used to rapidly determine 'interesting' events and keep full information from those collisions while discarding any other interactions. 

To successfully achieve these goals, the CMS detector must be able to rapidly identify outgoing particles from a large number of collisions, separate them into individual interactions, determine when physics processes of interest have occurred, and accurately measure and read out all final state particles in triggered events in order to allow for detailed reconstruction. 
To achieve this, the CMS detector consists of several subsystems located concentrically around the collision point. 
Starting from the center, particles first pass through the pixel and strip trackers, which precisely measure the position of charged particles to determine trajectories and vertex locations, as well as their momenta.
Next, particles energy the electromagnetic calorimeter (ECAL) which measures the energy of electrons, photons, and some mesons by completely stopping them through electromagnetic showers and measuring the resulting energy deposits.
After the ECAL, particles enter the hadronic calorimeter (HCAL), which measures neutral hadrons and other less interacting particles which may pass through the ECAL by having many layers of dense absorber to induce hardonic showers interleaved with scintillator layers to measure the resulting energy.
Finally, any remaining particles will pass into the muon chambers, which consist of cathode strips, resistive plates, and gas chambers which all track the motion of muons, as they are the only particles visible to our detector which are unlikely to be stopped in the calorimeters.
In addition to the detecting subsystems, the central feature of the CMS detector is a superconducting solenoid located between the HCAL and muon chambers, which produces a \SI{3.8}{\tesla} magnetic field along the axis of the beam. 
A steel return yoke is located within the muon chambers to capture the fringe field from the solenoid, producing a field in the opposite direction to aid with the measurement of muon momenta.
The large magnetic field provided by the solenoid allows for increased precision in particle momentum over shorter path lengths, and the relatively small size of the resulting detector and use of the return yoke for muon measurements give the detector its name.

To best describe particle momenta and take advantage of the nearly cylindrical symmetry of the detector, a special coordinate system is defined. 
The Z-axis is chosen to be oriented along the beam line, and the transverse distance from it is referred to as $\rho$. 
Instead of the polar angle $\theta$, we instead use the pseudorapidity $\eta$. 

\begin{equation}
    \label{eq:pseudo}
    \eta = - log \left[tan\left(\frac{\theta}{2}\right)\right]
\end{equation}

In special relativity, rapidity ($y$) is a measure of relativistic velocity with the feature that differences in rapidity are invariant under Lorentz transformations along the z-axis.
The definition of particle rapidity is shown in \cref{eq:rapidity}, where E is the particle's total energy and p$_z$ is its z-momentum.
\begin{equation}
    \label{eq:rapidity}
    y = \frac{1}{2} ~ln \left(\frac{E+p_z}{E-p_z}\right)
\end{equation}

Pseudorapidity is constructed to form an analogue of rapidity using only geometric information, and converges to match rapidity in the high-energy limit when particle mass is negligible.
Pseudorapidity is zero when perpendicular to the beam, and approaches positive and negative infinity along the z axis. 
By using $\eta$, selections using the angular separation of particles do not have strong dependence on longitudinal momentum and the overall distribution of outgoing collision products is nearly uniform instead of strongly peaked towards zero and $\pi$ as it is using $\theta$.  
The angular distance in this coordinate system is defined as $\Delta R = \sqrt{(\eta_1-\eta_2)^2+(\phi_1-\phi_2)^2}$.

\section{Tracker}
The innermost part of the detector is the CMS tracker, which uses silicon detectors arranged in pixels and strips to measure ionizing radiation from collision products. 
A full description of this detector can be found in Reference \cite{trackerTDR}.
When ionizing particles pass through the silicon sensors, they excite electron-hole pairs within the silicon, which are collected via an applied electic field in the material and produce an output current pulse in the readout electronics.
The positions of these deposits in the silicon are then used to reconstruct the tracks of charged particles.
Using tracks, the charge and momentum of particles can be determined through their curvature in the magnetic field and particle trajectories can be extrapolated to the collision point in order to group tracks from the same interaction into vertices. 
High positional resolution is very important for track reconstruction to resolve individual vertices and separate pileup interactions as well as to precisely determine the track curvature over the length of the tracker. 

The pixel detectors are positioned in the inner part of the CMS tracker, where the high particle density requires the best position resolution. 
The pixel detector was upgraded in 2017 \cite{pixelUpgrade}, and the upgraded version is used for the description here and all of the data samples used. 
The individual pixels are made of 100 $\times$ \SI{150}{\micro\meter} sensors which are placed in three disks in the endcap ($1.5<\eta<2.5$) region and four layers in the barrel ($\eta<1.5$) region, and have the short axis aligned in the radial direction so that measurements of curvature in $\phi$ from the magnetic field will have increased resolution from the smaller cell size. 
The resulution is further increased to sub-cell precision through charge-sharing from capacitive coupling to adjacent cells, resulting in final position resolution on the order of \SI{10}{\micro\meter} along the short axis and \SI{20}{\micro\meter} in the longitudinal direction.

In deeper parts of the tracking system, the increased spatial dimension and reduced particle density reduces the need for pixel resolution and instead silicon strips of up to \SI{10}{\centi\meter} in length are used. 
In the barrel region, 10 strip layers are present, reaching up to \SI{110}{\centi\meter} from the beam, and the strips are oriented slightly off-axis in alternating layers to increase the resolution in the z-direction. 
In the endcap, the strips are oriented perpendicularly to the beam in 12 disks up to a radial distance of \SI{270}{\centi\meter}.
Hits in the strip tracker have typical resolutions near \SI{200}{\micro\meter} in the Z-direction and \SI{20}{\micro\meter} in the radial direction.

The combination of pixel and strip trackers results in very good track reconstruction efficiency and position and momentum resolution.
Muon tracks are reconstructed with $>99\%$ efficiency over the entire $\eta$ range of the tracker, and vertices are reconstructed with resolutions near \SI{50}{\micro\meter}.
Because higher energy tracks have less curvature, the \pt resolution of tracks decreases as the \pt itself increases. Tracks with typical \pt and $\eta$ used in this analysis (\pt near \SI{100}{\giga\eV} and $1.4<\eta<2.4$) have resolutions near 4$\%$. 

\section{ECAL}

\section{HCAL}
The CMS hadronic calorimeter (HCAL) consists of (number) alternating layers of Brass absorber and scintillator. 

\subsection{The HCAL endcap upgrade}
In 2018, the 
\section{Muon Systems}