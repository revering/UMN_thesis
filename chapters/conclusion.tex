%%%%%%%%%%%%%%%%%%%%%%%%%%%%%%%%%%%%%%%%%%%%%%%%%%%%%%%%%%%%%%%%%%%%%%%%%%%%%%%%
% conclusion.tex:
%%%%%%%%%%%%%%%%%%%%%%%%%%%%%%%%%%%%%%%%%%%%%%%%%%%%%%%%%%%%%%%%%%%%%%%%%%%%%%%%
\chapter{Conclusiond and Outlook}
\label{conclusion_chapter}

A novel search for dark matter using the CMS detector via secondary interactions of muons with the detector has been presented.
In addition to setting a limit on the \dbrem interaction rate within CMS, the search establishes a novel technique of using standard model particles emitted in the initial collision to search for new physics in their later interactions, extending the CMS sensitivity to signatures with very long lifetimes or without visible particles in the final state.
The usage of low-level HCAL information also resulted in in-depth validation of the simulated HE response to muons, as well as the most precise alignment of HE to date.

There are many potential advancements that can be made for this search and others like it in the future. 
As the LHC transitions to higher luminosities, the huge number of particles produced and interactions recorded can be used to provide initial for these fixed-target type studies, and extend the range of initial particle states that can be used.
As this particular analysis is limited to 2018 data, even the expected luminosity from the current Run 3 data taking with the CMS detector will increase the available dataset size by a factor of six.

Additional improvements will also be available due to the upgrade of the CMS HCAL extending to the barrel in addition to the endcaps in Run 3, providing an additional factor of four increase in the selection efficiency of probe tracks.
On the analysis side, future studies may also take advantage of the CMS track trigger to select muons which may originate from $J/\psi$ particles instead of Z bosons, which has a much larger production cross section in CMS resulting in a significant increase in the number of available probes.

This analysis technique presents unique challenges alongside the potential sensitivity to new physics signatures. 
Initial particles must be identified without access to their full signature in the detector to allow for secondary interactions, and careful study of mis-identification rates is necessary to understand potential backgrounds from them.
These partial reconstructions and the signatures of secondary interactions are best observed by low-level detector information, which requires careful calibration and custom systematics.
Signal simulation for secondary interactions requires special consideration, and either full integration into the detector simulation or custom implementation.
The techniques presented in this thesis present possible solutions to these challenges, and can serve as a starting point for other analysis with similar constraints.

Despite the strong motivation for dark matter provided by astronomical observations, few strong constraints on its potential properties exist, and a very wide range of potential physics models could describe the observed phenomena.
To best cover the potential physics, searches must be broad and deep, using both the increased luminosity and phase space coverage presented by upgraded or new detectors as well as new analysis techniques to extend coverage to as many potential models as possible. 
Close collaboration between theory and experiment must be made to best identify signatures that may be visible in detectors, as well as interperet existing analysis results to constrain which models may remain.


%%%%%%%%%%%%%%%%%%%%%%%%%%%%%%%%%%%%%%%%%%%%%%%%%%%%%%%%%%%%%%%%%%%%%%%%%%%%%%%%

%%%%%%%%%%%%%%%%%%%%%%%%%%%%%%%%%%%%%%%%%%%%%%%%%%%%%%%%%%%%%%%%%%%%%%%%%%%%%%%%
